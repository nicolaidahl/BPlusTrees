%!TEX root = ./report.tex
\newpage
\appendix
\section{Results of Black Box Tests}
\label{app:blackboxtests}
\subsection{Input Models}
For these tests, we have several input models: One of them is a subset of a real IFC model, provided by Kaj Jørgensen. The rest of the test models have been created by Mathias Demant for our tests, and do not represent any real buildings. The files can be found in Prototype/Model2ModelWorkflows/NanoLabBuilding/ on the attached medium. The results can be seen in Figures \ref{fig:test1}, \ref{fig:test2}, \ref{fig:test3}, and \ref{fig:test4}.
\subsection{Tests}
For each case, we will perform the following tests between the two MWE2 workflows:
\subsubsection{Test 1:}
Modifying the metadata (name, description) for one of each of the three product types (wall, opening, flow segment).\newline\textbf{Expected result:} The output IFC model differs from the input IFC model in that it has altered metadata (name, description).
\subsubsection{Test 2:}
Modifying the placement of one of each of the three products types.\newline
\textbf{Expected result:} A new IFCLocalPlacement object now exists in the IFC output model with the modified metadata. 
\subsubsection{Test 3:}
Modifying the list of walls in a opening element.\newline
\textbf{Expected result:} A new IFCRelVoidElement is present in the output IFC model, compared to the input mode, linking the wall and opening.
\subsubsection{Test 4:}
Deleting a Pipe.\newline
\textbf{Expected result:} The output IFC model no longer contains the deleted flow segment. Any created orphans should be garbage collected.
\subsubsection{Test 5:}
Adding an Opening.\newline
\textbf{Expected result:} The output IFC model should now contain the created opening, along with any new objects referenced by the opening (LocalPlacement, Axis2Placement3d)
\subsubsection{Test 6:}
Adding one opening, deleting one flow segment, and modifying metadata for one wall.\newline
\textbf{Expected result:} The output IFC model should now contain the created opening (along with referenced objects), the flow segment should be deleted (along with orphans), and the metadata for the wall should be updated.
\begin{figure}
    \centering
        \centerline{\includegraphics[width=150mm]{images/Test1.pdf}}
    \caption{}
    \label{fig:test1}
\end{figure}
\begin{figure}
    \centering
        \centerline{\includegraphics[width=150mm]{images/Test2.pdf}}
    \caption{}
    \label{fig:test2}
\end{figure}
\begin{figure}
    \centering
        \centerline{\includegraphics[width=150mm]{images/Test3.pdf}}
    \caption{}
    \label{fig:test3}
\end{figure}
\begin{figure}
    \centering
        \centerline{\includegraphics[width=150mm]{images/Test4.pdf}}
    \caption{}
    \label{fig:test4}
\end{figure}


\section{Automated Tests}
\label{app:automatedtests}
The workflow components used for tests can be found in the Eclipse project under: Model2ModelWorkflows/src/tests. These components work as a part of two test workflows similar to the ones used in the actual prototype. The only difference is that test components have been inserted after extraction and transformation components in the test workflow. The IFC2Pipes.mwe2 workflow is thus tested by the IFC2PipesTest.mwe2 workflow and the Pipes2IFC.mwe2 workflow is tested by the Pipes2IFCTest.mwe2 workflow.

\subsubsection{Testing Element Extraction}
The TestIFCExtractor workflow component verifies that the extraction process executed with a correct result. On any input building it checks that the number of extracted IfcProducts corresponds to the number of products that should be extracted by looking directly at the ifcXML. For each of the extracted product elements it tests for a valid globalId and Id. The tests have been run with the Merging\_{}test\_{}small model, only as this is the most realistic model. All assertions succeed in this test.

\subsubsection{Testing IFC to Pipes Model Transformation}
The following consistency verification between elements in the IFC and Pipes models are carried out:

\begin{itemize}
	\item IfcFlowSegment <-> FlowSegment
	\item IfcOpening <-> Opening
	\item IfcWall <-> Wall
	\item IfcLocalPlacement <-> LocalPlacement
	\item IfcAxis2Placement3d <-> Axis2Placement3D
	\item IfcDirection <-> Direction
\end{itemize}

By going through all extracted IFC products after the model to model transformation and comparing them to the elements of the Pipes model, the TestTransformers test component automates transformation testing of the subdomain. All consistency tests on Merging\_{}test\_{}small pass.

\subsubsection{Testing Pipes-to-IFC Update Routine}
The TestTransformer test component is used for the purpose of testing the Pipes-to-IFC update routine as well. The same consistency checks are performed after the Pipes2IFCTransformer, which executes the update routine. The same element relations as above are tested. All consistency tests on Merging\_{}test\_{}small pass.





