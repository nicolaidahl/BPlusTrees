\documentclass[oribibl]{llncs}
\pagestyle{headings} %page numbers
\usepackage{makeidx}  % allows for indexgeneration
\usepackage[utf8]{inputenc}
\usepackage{pdfpages}
\usepackage{float}
\usepackage{enumitem}
\usepackage[T1]{fontenc} % Fixed missing font warning for \maketitle
\usepackage{url}
\usepackage{todonotes} % remove later - this is for TODO notes
%\usepackage{amsmath,amssymb}  We don't use this?
\usepackage{coqdoc}
\usepackage{url}

\hypersetup{%
  citecolor=black
}

\bibliographystyle{plain}

\begin{document}
\mainmatter
\title{Formal Verification of B+ Trees}
\author{Nicolai Dahl Blicher-Petersen \and Christian Harrington \and Morten Fangel Jensen \\
\email{\{ndbl, cnha, mfan\}@itu.dk}}
\institute{IT University of Copenhagen, Rued Langgaards Vej 7, 2300 Copenhagen S, Denmark}

\maketitle

\begin{abstract}
The B+ tree data structure is a balanced, n-ary tree, used most often in database and file system storage. It is known for its high fanout, thereby minimizing the number of costly I/O operations. Using the Coq interactive proof assistant we define an inductive data type describing a valid B+ tree, used to formally verify properties of the search and insert operations of the data structure. Furthermore, we describe the need for an added level of indirection in the form of an intermediate proposition when proving relations between the insert and search operations.

\keywords{B+ tree, Coq, Gallina}
\end{abstract}

%!TEX root = ../BPlusTree-report.tex
\section{Introduction}
\label{sec:Introduction}
% Notes:
% Implement BPlusTree in Gallina
% What are we going to prove?
% We use Coq
In the following report we discuss the implementation and formal proof of an inherently imperative data structure, B+ tree, in a pure functional language, Galina. 
\paragraph{}
First, in Section \ref{sec:Background} we explain the data structure in question. Section \ref{sec:ProblemAnalysis} gives a problem analysis focused on the implementation of this data structure with its core operations, when proving these is an obligation. The various strategies and difficulties when proving the main operations are covered in Section \ref{sec:ProofRealization}. Finally, Section \ref{sec:Conclusion} concludes the report.
%!TEX root = ../BPlusTree-report.tex
\section{Background}
\label{sec:Background}
% Notes:
% What is a bplustree:
%   - Inherently imperative data structure
%   - Suboptimal implementation
%     - Running time of optimal implementation
%     - Running time of our implementation
%       - Mention how one could make it optimal.
%       - A tree in a tree in a tree, dawg
%     - Running time out of scope

\subsection{Gallina}
Gallina is a purely functional language\todo{ref}, which is used by the Coq interactive proof assistant. It is very lean, and does not include non-functional data structures, such as arrays\todo{ref}.

\subsection{B+ tree}
The B+ tree is a n-ary, self-balancing, tree data structure\todo{ref}, similar to a B-tree\todo{ref}. It is composed of a root, nodes, and leaves. The root may be a leaf or a node. Nodes hold pairs of keys and pointers, $(k, p)$. $p$ points to either a node or leaf that holds the values over $k$, but below the key of the next pair. Leaves hold pairs of keys and values, $(k, v)$. In this project keys are always natural numbers, while values can have any type, denoted by $X$. In this project duplicate keys are not allowed.

\begin{figure}
 \centering
   \includegraphics[width=90mm]{diagrams/BPlusTree.pdf}
 \caption{An example of a B+ tree with $b=1$. The bottom row contains leaves, with values, in this case characters, in the dashed boxes.}
 \label{fig:bplustree}
\end{figure}

For a given B+ tree, a branching factor $b$ determines the capacity of nodes and leaves\todo{ref}. A node must have at least $b+1$ children, and at most $2b+1$. A leaf must have between $b$ and $2b$ values. This rule is relaxed for the root node, which must have between 2 and $2b+1$ children. If the root is a leaf, it must have between 0 and $2b$ values.

\paragraph{}
In our implementation, a node is a list of key-pointer $(k, p)$ pairs where $p$ points to a child tree. A leaf is a list of key-value $(k, v)$ pairs. We support three operations, which we will refer to as our primary operations. These are: insert, search and height. The theory behind the height function is trivial, and will not be explained in this section, but the theory behind the insertion and search functions require some explanation. Beyond these primary operations, we have also implemented a simple in-order traversal, for which it is worth noting that pointers between leafs for a speed-up in sequential access (range queries), which some implementations of B+ trees include, has not been implemented. The in-order traversal will not be mentioned further in this report.

\subsubsection{Search}
The search function takes two arguments: a search key $sk$, and a tree $t$, to search in. Tree can be either a node or a leaf. If the the tree is a node, the function recurses through the $(k, p)$ list, until it finds the pair $(k, p)$, value where $k \le sk$, and $sk < k1$, where $k1$ is the following pair $(k1, p1)$ in the list. The function then recursively calls itself with the subtree $p$. Once the function reaches a leaf, the $(k, v)$ list is searched, often by doing a binary search.

\subsubsection{Insert}
The insert function takes two arguments: a key-value pair $(k, v)$ and a tree $t$ to insert the pair into. The function starts at the root of the tree, and recursive. The pair is then inserted into the leaf at the correct position. If the insertion results in the leaf having more than $2b$ values, the leaf must be split. This is called overflow. This is done by splitting the leaf in half, and inserting a pointer to the new leaf in the parent node. If the parent node now has more than $2b+1$ children, it most also be split, following the same pattern. This can continue all the way to the root, which in turn can be split. If this happens, a new root node is created, as a parent to the old root, and the height of the tree increases. Thus, a B+ tree can be seen to grow up from the leaves.

%!TEX root = ../BPlusTree-report.tex
\section{Problem Analysis}
\label{sec:ProblemAnalysis}
% Notes:
% Relevant constructs
%   - bplustree
%   - insert
%   - search 
%   - height
%   - deletion
% We want to prove:
%   - Insert works
%     - Inductive data types
%       - valid_bplus_tree
%       - appears_in_kvl
%       - appears_in_tree
%       - kvl_sorted
%     - Works under these assumptions...
%       -Valid bplustree
%       - Insertion preserves tree
%   - Search works
To implement B+ trees in Gallina, several different components have to be implemented. Most importantly, we must specify an inductive data type that describes a B+ tree, which can be seen in Figure \ref{fig:inductive_data_type}.

\begin{figure}
\centering
\begin{coqdoccode}
\coqdockw{Inductive} \coqdocvar{bplustree} (\coqdocvar{b}: \coqdocvar{nat}) (\coqdocvar{X}:\coqdockw{Type}) : \coqdockw{Type} :=\coqdoceol
\coqdocindent{1.00em}
\ensuremath{|} \coqdocvar{bptLeaf} : \coqdocvar{list} (\coqdocvar{nat} \ensuremath{\times} \coqdocvar{X}) \ensuremath{\rightarrow} \coqdocvar{bplustree} \coqdocvar{b} \coqdocvar{X}\coqdoceol
\coqdocindent{1.00em}
\ensuremath{|} \coqdocvar{bptNode} : \coqdocvar{list} (\coqdocvar{nat} \ensuremath{\times} (\coqdocvar{bplustree} \coqdocvar{b} \coqdocvar{X})) \ensuremath{\rightarrow} \coqdocvar{bplustree} \coqdocvar{b} \coqdocvar{X}.\coqdoceol
\end{coqdoccode}
\caption{Inductive data type for B+ tree.}
\label{fig:inductive_data_type}
\end{figure}

\paragraph{}
Now that we have a inductive data type for B+ trees, we can start working on proofs. For this project, our focus will be on proving our implementations of the $insert$ function and the $search$ function.

\subsubsection{Insert}

\paragraph{}
\todo{Write about valid bplus}

\paragraph{}
\todo{Write about appears in...}

\paragraph{}

\subsubsection{Search}
\todo{Write about sorted}
%!TEX root = ../BPlusTree-report.tex
\section{Proof Realization}
\label{sec:ProofRealization}
% Notes:
% bplustree inductive data type
%   - Why not have a start pointer/end pointer?
% insert/search
%   - Use appears_in instead of search(insert)
% insert
%   - mutually recursive
%   - Kopitiam cannot handle large proof assumptions
%   -counter
In this section we will examine how the proof of InsertSearchWorks was realized. Firstly, we give an overview of the supporting inductive data types as well as the overall strategy we have used to approach this proof obligation. Secondly, we break the problem down into the proof obligations related to search and insert, respectively. \todo{Maybe rewrite this little intro}

\subsection{Proofs about sorting}
A vital aspect of B+ trees is that all of the key-point and key-value lists in nodes and leaves are sorted by the key. So the first proposition we defined was $kvl\_sorted$, which is only applicable to such sorted lists. The proposition is reproduced in Figure \ref{fig:kvl_sorted}.

\begin{figure}
  \begin{coqdoccode}
  \coqdocnoindent
  \coqdockw{Inductive} \coqdocvar{kvl\_sorted} \{\coqdocvar{X}: \coqdockw{Type}\}: \coqdocvar{list} (\coqdocvar{nat} \ensuremath{\times} \coqdocvar{X}) \ensuremath{\rightarrow} \coqdockw{Prop} :=\coqdoceol
  \coqdocindent{1.00em}
  \ensuremath{|}
  \coqdocvar{kvl\_sorted\_0}: \coqdocvar{kvl\_sorted} []\coqdoceol
  \coqdocindent{1.00em}
  \ensuremath{|} \coqdocvar{kvl\_sorted\_1}: \coqdockw{\ensuremath{\forall}} (\coqdocvar{n}: \coqdocvar{nat}) (\coqdocvar{x}: \coqdocvar{X}), \coqdoceol
  \coqdocindent{8.00em}
  \coqdocvar{kvl\_sorted} [(\coqdocvar{n}, \coqdocvar{x})]\coqdoceol
  \coqdocindent{1.00em}
  \ensuremath{|} \coqdocvar{kvl\_sorted\_cons}: \coqdockw{\ensuremath{\forall}} (\coqdocvar{n1} \coqdocvar{n2}: \coqdocvar{nat}) (\coqdocvar{x1} \coqdocvar{x2}: \coqdocvar{X}) (\coqdocvar{lst}: \coqdocvar{list} (\coqdocvar{nat} \ensuremath{\times} \coqdocvar{X})), \coqdoceol
  \coqdocindent{8.00em}
  \coqdocvar{kvl\_sorted} ((\coqdocvar{n2},\coqdocvar{x2})::\coqdocvar{lst}) \ensuremath{\rightarrow} \coqdoceol
  \coqdocindent{8.00em}
  \coqdocvar{blt\_nat} \coqdocvar{n1} \coqdocvar{n2} = \coqdocvar{true} \ensuremath{\rightarrow}\coqdoceol
  \coqdocindent{8.00em}
  \coqdocvar{kvl\_sorted} ((\coqdocvar{n1},\coqdocvar{x1})::(\coqdocvar{n2},\coqdocvar{x2})::\coqdocvar{lst}).\coqdoceol
  \end{coqdoccode}
  \caption{Our proposition about sorting}
  \label{fig:kvl_sorted}
\end{figure}

\paragraph{}
Because almost all of our proofs entails manipulating sorted list, we first built up a extensive set of lemmas and theorems detailing how the proposition behaves when the list is changed --- e.g. if you remove the head of the list, the remainder is still sorted. We have reproduced a few of these behaviors in Figure \ref{fig:key_sorting_lemmas}. The theorem $insert\_preserves\_sort$ is probably the one with most direct impact to the rest of the proofs. This is the lemma that allows us to insert new items into a key-value or key-pointer list and know that the list continues to be sorted. $sort\_ignores\_values$ simply confirms that our sorting is only concerned with the keys in the list, as we can swap out one key for another without impacting the validity of the proposition. $list\_tail\_is\_sorted$ is a very useful lemma when dealing with induction over lists, because it quickly allows us to pop the head element of a list and still know that the resulting list is sorted.

\begin{figure}
  \begin{coqdoccode}
  \coqdocnoindent
  \coqdockw{Lemma} \coqdocvar{sort\_ignores\_value} : \coqdockw{\ensuremath{\forall}} (\coqdocvar{X}: \coqdockw{Type}) (\coqdocvar{k}: \coqdocvar{nat}) (\coqdocvar{v1} \coqdocvar{v2}: \coqdocvar{X}) (\coqdocvar{l}: \coqdocvar{list} (\coqdocvar{nat} \ensuremath{\times} \coqdocvar{X})),\coqdoceol
  \coqdocindent{1.00em}
  \coqdocvar{kvl\_sorted} ((\coqdocvar{k},\coqdocvar{v1})::\coqdocvar{l}) \ensuremath{\rightarrow} \coqdocvar{kvl\_sorted}((\coqdocvar{k}, \coqdocvar{v2})::\coqdocvar{l}).\coqdoceol
  \coqdocemptyline
  \coqdocnoindent
  \coqdockw{Lemma} \coqdocvar{list\_tail\_is\_sorted} : \coqdockw{\ensuremath{\forall}} (\coqdocvar{X}: \coqdockw{Type}) (\coqdocvar{l}: \coqdocvar{list} (\coqdocvar{nat} \ensuremath{\times} \coqdocvar{X})) (\coqdocvar{k}: \coqdocvar{nat}) (\coqdocvar{v}: \coqdocvar{X}),\coqdoceol
  \coqdocindent{1.00em}
  \coqdocvar{kvl\_sorted} ((\coqdocvar{k},\coqdocvar{v})::\coqdocvar{l}) \ensuremath{\rightarrow} \coqdocvar{kvl\_sorted} \coqdocvar{l}.\coqdoceol
  \coqdocemptyline
  \coqdocnoindent
  \coqdockw{Lemma} \coqdocvar{kvl\_sorted\_key\_across\_app} : \coqdockw{\ensuremath{\forall}} (\coqdocvar{X}: \coqdockw{Type}) (\coqdocvar{l1} \coqdocvar{l2}: \coqdocvar{list} (\coqdocvar{nat} \ensuremath{\times} \coqdocvar{X})) (\coqdocvar{k1} \coqdocvar{k2}: \coqdocvar{nat}) (\coqdocvar{v1} \coqdocvar{v2}: \coqdocvar{X}),\coqdoceol
  \coqdocindent{1.00em}
  \coqdocvar{kvl\_sorted}((\coqdocvar{k1}, \coqdocvar{v1})::\coqdocvar{l1} ++ (\coqdocvar{k2}, \coqdocvar{v2})::\coqdocvar{l2}) \ensuremath{\rightarrow} \coqdocvar{k1} < \coqdocvar{k2}.\coqdoceol
  \coqdocemptyline
  \coqdocnoindent
  \coqdockw{Theorem} \coqdocvar{insert\_preserves\_sort} : \coqdockw{\ensuremath{\forall}} (\coqdocvar{X}: \coqdockw{Type}) (\coqdocvar{l}: \coqdocvar{list} (\coqdocvar{nat} \ensuremath{\times} \coqdocvar{X})) (\coqdocvar{k}: \coqdocvar{nat}) (\coqdocvar{v}: \coqdocvar{X}),\coqdoceol
  \coqdocindent{1.00em}
  \coqdocvar{kvl\_sorted} \coqdocvar{l} \ensuremath{\rightarrow} \coqdocvar{kvl\_sorted} (\coqdocvar{insert\_into\_list} \coqdocvar{k} \coqdocvar{v} \coqdocvar{l}).\coqdoceol
  \coqdocemptyline
  \end{coqdoccode}
  \caption{Key lemmas and theorems about sorting}
  \label{fig:key_sorting_lemmas}
\end{figure}

\subsection{Intermediate proposition}
\label{intermediate_prop}

We have chosen to prove the correctness of our implementation using a added level of indirection. Instead of directly proving that $search~k~(insert~k~v~ tree) = Some~v$, we are instead proving after inserting into a tree, we know that the tree has a certain property: that it contains the inserted key. Likewise we prove that if this property holds for a tree, then $search$ can retrieve the item. The implications that must hold using our intermediate propositions can be seen in Equation \ref{intermediate_model}.

\begin{equation}
  insert~k~v~t = t' \rightarrow kw\_appears\_in\_tree~k~v~t' \rightarrow search~k~t' = Some~v
  \label{intermediate_model}
\end{equation}

This indirection allows us to prove that $search$ works independently from proving that $insert$ works.

\subsubsection{Reasoning about contents}
To verify that our solution can search and insert into both leaves and entire trees, we designed two propositions that allows us to reason about the content of leaves and trees. The two propositions can be seen represented in Figure \ref{fig:aik_and_ait}. 

\begin{figure}
\begin{coqdoccode}
  \coqdocnoindent
  \coqdockw{Inductive} \coqdocvar{kv\_appears\_in\_kvl} \{\coqdocvar{X}:\coqdockw{Type}\} (\coqdocvar{sk}: \coqdocvar{nat}) (\coqdocvar{sv}: \coqdocvar{X}) : \coqdocvar{list} (\coqdocvar{nat} \ensuremath{\times} \coqdocvar{X}) \ensuremath{\rightarrow} \coqdockw{Prop} :=\coqdoceol
  \coqdocindent{1.00em}
  \ensuremath{|} \coqdocvar{kv\_aik\_here}: \coqdockw{\ensuremath{\forall}} \coqdocvar{l}, ~~~~~~\coqdocvar{kv\_appears\_in\_kvl} \coqdocvar{sk} \coqdocvar{sv} ((\coqdocvar{sk}, \coqdocvar{sv})::\coqdocvar{l})\coqdoceol
  \coqdocindent{1.00em}
  \ensuremath{|} \coqdocvar{kv\_aik\_later}: \coqdockw{\ensuremath{\forall}} \coqdocvar{k} \coqdocvar{v} \coqdocvar{l},	\coqdocvar{kv\_appears\_in\_kvl} \coqdocvar{sk} \coqdocvar{sv} \coqdocvar{l} \ensuremath{\rightarrow} \coqdoceol
  \coqdocindent{11em} \coqdocvar{kv\_appears\_in\_kvl} \coqdocvar{sk} \coqdocvar{sv} ((\coqdocvar{k}, \coqdocvar{v})::\coqdocvar{l}).\coqdoceol
  \coqdocemptyline
  \coqdocnoindent
  \coqdockw{Inductive} \coqdocvar{kv\_appears\_in\_tree} \{\coqdocvar{X}:\coqdockw{Type}\} \{\coqdocvar{b}: \coqdocvar{nat}\} (\coqdocvar{sk}: \coqdocvar{nat}) (\coqdocvar{sv}: \coqdocvar{X}) : \coqdocvar{bplustree} \coqdocvar{b} \coqdocvar{X} \ensuremath{\rightarrow} \coqdockw{Prop} :=\coqdoceol
  \coqdocindent{1.00em}
  \ensuremath{|} \coqdocvar{kv\_ait\_leaf}: \coqdockw{\ensuremath{\forall}} \coqdocvar{l},\coqdoceol
  \coqdocindent{8.00em}
  \coqdocvar{kv\_appears\_in\_kvl} \coqdocvar{sk} \coqdocvar{sv} \coqdocvar{l} \ensuremath{\rightarrow} \coqdoceol
  \coqdocindent{8.00em}
  \coqdocvar{kv\_appears\_in\_tree} \coqdocvar{sk} \coqdocvar{sv} (\coqdocvar{bptLeaf} \coqdocvar{b} \coqdocvar{X} \coqdocvar{l})\coqdoceol
  \coqdocindent{1.00em}
  \ensuremath{|} \coqdocvar{kv\_ait\_node\_last}: \coqdockw{\ensuremath{\forall}} \coqdocvar{k1} \coqdocvar{k2} \coqdocvar{v1} \coqdocvar{v2}, \coqdoceol
  \coqdocindent{8.00em}
  \coqdocvar{kv\_appears\_in\_tree} \coqdocvar{sk} \coqdocvar{sv} \coqdocvar{v2} \ensuremath{\rightarrow} \coqdocvar{k2} \ensuremath{\le} \coqdocvar{sk} \ensuremath{\rightarrow}\coqdoceol
  \coqdocindent{8.00em}
  \coqdocvar{kv\_appears\_in\_tree} \coqdocvar{sk} \coqdocvar{sv} (\coqdocvar{bptNode} \coqdocvar{b} \coqdocvar{X} [(\coqdocvar{k1}, \coqdocvar{v1}), (\coqdocvar{k2}, \coqdocvar{v2})])\coqdoceol
  \coqdocindent{1.00em}
  \ensuremath{|} \coqdocvar{kv\_ait\_node\_here}: \coqdockw{\ensuremath{\forall}} \coqdocvar{k1} \coqdocvar{k2} \coqdocvar{v1} \coqdocvar{v2} \coqdocvar{l}, \coqdoceol
  \coqdocindent{8.00em}
  \coqdocvar{kv\_appears\_in\_tree} \coqdocvar{sk} \coqdocvar{sv} \coqdocvar{v1} \ensuremath{\rightarrow} \coqdocvar{k1} \ensuremath{\le} \coqdocvar{sk} \ensuremath{\land} \coqdocvar{sk} < \coqdocvar{k2} \ensuremath{\rightarrow}\coqdoceol
  \coqdocindent{8.00em}
  \coqdocvar{kv\_appears\_in\_tree} \coqdocvar{sk} \coqdocvar{sv} (\coqdocvar{bptNode} \coqdocvar{b} \coqdocvar{X} ((\coqdocvar{k1}, \coqdocvar{v1})::(\coqdocvar{k2}, \coqdocvar{v2})::\coqdocvar{l}))\coqdoceol
  \coqdocindent{1.00em}
  \ensuremath{|} \coqdocvar{kv\_ait\_node\_later}: \coqdockw{\ensuremath{\forall}} \coqdocvar{x} \coqdocvar{k1} \coqdocvar{k2} \coqdocvar{v1} \coqdocvar{v2} \coqdocvar{l},\coqdoceol
  \coqdocindent{8.00em}
  \coqdocvar{kv\_appears\_in\_tree} \coqdocvar{sk} \coqdocvar{sv} (\coqdocvar{bptNode} \coqdocvar{b} \coqdocvar{X} ((\coqdocvar{k1}, \coqdocvar{v1})::(\coqdocvar{k2}, \coqdocvar{v2})::\coqdocvar{l})) \ensuremath{\rightarrow} \coqdoceol
  \coqdocindent{8.00em}
  \coqdocvar{k1} \ensuremath{\le} \coqdocvar{sk} \ensuremath{\rightarrow}\coqdoceol
  \coqdocindent{8.00em}
  \coqdocvar{kv\_appears\_in\_tree} \coqdocvar{sk} \coqdocvar{sv} (\coqdocvar{bptNode} \coqdocvar{b} \coqdocvar{X} (\coqdocvar{x}::(\coqdocvar{k1}, \coqdocvar{v1})::(\coqdocvar{k2}, \coqdocvar{v2})::\coqdocvar{l})).\coqdoceol
  \end{coqdoccode}
\caption{Inductive propositions for reasoning about contents}
\label{fig:aik_and_ait}
\end{figure}

\paragraph{}
We can reason about leaves using the $kw\_appears\_in\_kvl$ proposition on the leaves key-value pairs. If a key is present in the list, then the proposition holds. Likewise the list can not contain a given key if the proposition does not hold. $kw\_appears\_in\_tree$ conveys the same properties for entire B+ trees --- if and only if a key is present the tree, the proposition holds. Often it can be beneficial to reason about only whether or not a given key exists, but not caring about the value. For this reason we also have the $appears\_in\_kvl$ and $appears\_in\_tree$ proposition. The relationship between two two are: $kv\_appears\_in\_kvl~k~v~l \rightarrow appears\_in\_kvl~k~l$. The reason we are interested in knowing if only a key exists, is so we can know if a insert will cause a overflow.

\paragraph{}
$kw\_appears\_in\_kvl$ is inductively defined over the key-value lists and has just two constructors: Either the key must appear at the head of the list or the key must appear later in the list.

\paragraph{}
Because of the nature of B+ trees, the $kw\_appears\_in\_tree$ proposition is somewhat more complicated. It has a single constructor for leaves that simply requires that the $kw\_appears\_in\_kvl$ must hold for the leaf. For nodes, however, we must have 3 different constructors to ensure that the proposition only holds if the proposition also holds for the correct subtree.

\subsection{Reasoning about search}
We want to prove that $search$ actually performs like we expect it to. So if a key is present in a tree, we expect $search$ to find a value. Likewise we expect $search$ to not find anything when the key is not present in the tree.
Put more succinctly, we must prove that $kw\_appears\_in\_tree~k~v~tree \rightarrow search~k~tree = Some~v$ and $\lnot appears\_in\_tree~k~tree \rightarrow search~k~tree = None$.

\begin{figure}
  \begin{coqdoccode}
  \coqdocnoindent
  \coqdockw{Theorem} \coqdocvar{appears\_search\_works} : \coqdockw{\ensuremath{\forall}} (\coqdocvar{b}: \coqdocvar{nat}) (\coqdocvar{X}: \coqdockw{Type}) (\coqdocvar{t}: \coqdocvar{bplustree} \coqdocvar{b} \coqdocvar{X}) (\coqdocvar{k}: \coqdocvar{nat}),\coqdoceol
  \coqdocindent{1.00em}
  \coqdocvar{valid\_bplustree} \coqdocvar{b} \coqdocvar{X} \coqdocvar{t} \ensuremath{\rightarrow} \coqdoceol
  \coqdocindent{1.00em}
  \coqdocvar{kw\_appears\_in\_tree} \coqdocvar{k} \coqdocvar{v} \coqdocvar{t} \ensuremath{\rightarrow} \coqdoceol
  \coqdocindent{1.00em}
 \coqdocvar{search} \coqdocvar{k} \coqdocvar{t} = \coqdocvar{Some}(\coqdocvar{v}).\coqdoceol
  \end{coqdoccode}
  \caption{The theorem that states how $search$ must find a item that appears in the tree.}
  \label{fig:search_works}
\end{figure}

\paragraph{}
To prove the theorem shown in Figure \ref{fig:search_works} we first proved that our implementation could find items in a leaf using the $appears\_in\_kvl$ proposition. Because $appears\_in\_kvl$ is defined inductively over the same list that $search\_leaf$ is recursively defined over, this proof is trivial.

\paragraph{}
For proving the $appears\_search\_works$, we needed to argue that $search'$ has the same properties that we want $search$ to have. Because the definition of $search$ is a simple call to $search'$, the interesting proof is $appears\_search'\_works$. Like mentioned in the problem analysis, a induction over the counter argument of $search'$ it is equivalent to performing a induction over the height of the tree. By isolating the the subtree that $find\_subtree$ finds and applying the induction hypothesis on this subtree, we know that $search'$ will find the key in the subtree. Since $search'$ does not perform any modifications we can conclude that if it works for the subtree that $find\_subtree$, it works for the parent too. 

\subsubsection{A subtree must be found}
A important aspect of our recursion over the height of the subtree, is that we use $find\_subtree$ to identify which subtree to recurse into. So if $find\_subtree$ doesn't find a subtree, we have nothing to apply the induction hypothesis on, and we can not prove the theorem. Hence we must require that
$valid\_bplustree~b~X~t \leftarrow \forall~sk, \exists~subtree, find\_subtree~ sk~t = Some~subtree$.

\paragraph{}
Identifying a subtree is a matter of finding the two consecutive keys where the search-key falls within the range of those two keys, that is $k_i \le sk < k_{i+1}$. In a text-book implementation implementation of of B+ trees you also have the two cases $sk < k_0$ and $k_{n} \le sk$ where n is the number of keys. This ensures that no matter what keys exists, you can always use one of the 3 cases to find the range belonging to a subtree. But because we have simplified the data type for nodes, where we have a key for the first pointer as well, we can no longer use the the rule $sk < k_0$. So if a list key-pointer pair where the first key is bigger than the search-key were to be allowed, $find\_subtree$ can fail to find a subtree. This is why $valid\_bplustree$ states that the first key in all key-pointer lists must be 0. This serves to introduce the equivalent to the rule $sk < k_0$, because we now have $k_0 = 0 \le sk < k_1$. With this, we were able to prove that $find_subtree$ will always find a subtree to recurse into, and thus that $insert$ can traverse down to the leaf and find the key in question.

\subsection{Proving that insertion implies $kw\_appears\_in\_tree$}

In order to use our model where we use a intermediate proposition in our proofs, we must be able to prove that the $kw\_appears\_in\_tree$ proposition holds after inserting a key-value pair into any given, valid, tree. The theorem can be seen reproduced in Figure \ref{insert_works}.

\begin{figure}
  \begin{coqdoccode}
  \coqdocnoindent
  \coqdockw{Theorem} \coqdocvar{insert\_works} : \coqdockw{\ensuremath{\forall}} \{\coqdocvar{X}: \coqdockw{Type}\} \{\coqdocvar{b}: \coqdocvar{nat}\} (\coqdocvar{t}: \coqdocvar{bplustree} \coqdocvar{b} \coqdocvar{X}) (\coqdocvar{k}: \coqdocvar{nat}) (\coqdocvar{v}: \coqdocvar{X}),\coqdoceol
  \coqdocindent{1.00em}
  \coqdocvar{valid\_bplustree} \coqdocvar{b} \coqdocvar{X} \coqdocvar{t} \ensuremath{\rightarrow} \coqdoceol
  \coqdocindent{1.00em}
  \ensuremath{\lnot}\coqdocvar{appears\_in\_tree} \coqdocvar{k} \coqdocvar{t} \ensuremath{\rightarrow} \coqdoceol
  \coqdocindent{1.00em}
  \coqdocvar{kw\_appears\_in\_tree} \coqdocvar{k} \coqdocvar{v} (\coqdocvar{insert} \coqdocvar{k} \coqdocvar{v} \coqdocvar{t}).\coqdoceol
  \end{coqdoccode}
  \caption{The theorems stating that $insert$ implies $kw\_appears\_in\_tree$}
  \label{fig:insert_works}
\end{figure}

Just like with $search$, our definition of $insert$ is simple definition that calls $insert'$. To prove $insert'$ we again use the tactic of performing induction on our introduced $counter$ argument. Once we have identified the the subtree where the insertion must happen, we can use the induction hypothesis to prove that the value was correctly inserted into the child. But the task of proving that the child is correctly updated in the tree, and that this implies that the value now exists in the parent is non-trivial. The reason why is that there are multiple different cases that can happen. 

When we only know that the item was correctly inserted into a child that may or may not have overflowed, we have all of the following options for what can happen. In reality we need to handle all of the cases twice: Once for when the child is a leaf and once for when the child is a node.

\begin{enumerate}
  \item The child did not overflow, and hence the parent will not overflow. 
  \item The child overflowed.
  \begin{enumerate}
    \item The item was inserted into the regular child.
    \begin{enumerate}
      \item The parent had room for the overflow.
      \item The overflow caused the parent to overflow too.
      \begin{enumerate}
        \item The item ended up in the regular parent
        \item The item ended up in the overflow parent
      \end{enumerate}
    \end{enumerate}
    \item The item was inserted into the overflow child.
    \begin{enumerate}
      \item The parent had room for the overflow.
      \item The overflow caused the parent to overflow too.
      \begin{enumerate}
        \item The item ended up in the regular parent
        \item The item ended up in the overflow parent
      \end{enumerate}
    \end{enumerate}
  \end{enumerate}
\end{enumerate}

A lot of the cases are very similar, and we can see that 2.(a).ii and 2.(b).ii are identical. It was hence possible to construct some lemmas that allows reuse between cases but despite this the sheer amount of cases made the theorem rather complicated and incomprehensible. 

\subsection{Tying it together}
Fill me in.

%!TEX root = ../BPlusTree-report.tex
\section{Evaluation}
\label{sec:Evaluation}
Formally verifying a non-trivial data type like B+ trees has proven to be a difficult task. In this section we evaluate four central points regarding our work process and our InsertSearchWorks proof.

\subsubsection{Insert Into Leaf}
Our strategy for proving the $insert\_impl\_appears$ proof, was to prove the base case separately. The base case would be inserting into a tree with $height = 0$, or in other words a leaf. So the initial lemma we sat out to prove was $insert\_leaf\_impl\_appears$. But because we had not even attempted verifying the behaviors of inserting in nodes, we had some incorrect assumptions about what we should prove. One problem was that we didn't make the verified statements very strong, so when we later had to apply the lemma in $insert\_impl\_appears$ we had to establish a lot of context before we could apply the lemma. When we later sat out to verify $insert'$, we improved this aspect and verified a much stronger statement.

\subsubsection{Unnecessary Assumption}

Another shortcoming in our initial assumptions when verifying $insert\_leaf\_impl\_appears$ was that the key was not present before insertion. As a result, our verification of insertion into leafs has the requirement that $\lnot appears\_in\_kvl$ must hold. Because we rely on the verification for leafs when verifying insertion in trees, this assumption bubbles up and adds the requirement that $\lnot appears_in_tree$ must hold before inserting. In retrospect this assumption is unnecessary and results we have only a slightly weaker verification. We have only verified that inserting a new value works, not that overwriting an existing key works too. 

\subsubsection{Higher Priority for Validity Preservation Proof}

With a better overview we might have chosen to prove $insert\_preserves\_validity$ first. Our $insert\_search\_works$ proof relies on insertion preserving the validity, so without verification of this preservation we can not create a complete verification that inserted items can be found. The way we have structured our proofs, is only in the final lemma we require this preservation, so our two intermediate theorems are fully verified.

The reason why we chose to verify $insert\_search\_works$ first, that that we felt that a such verification was more telling about our actual implementation rather than proving that we maintained a inductively defined proposition. The verification of insertion and search relies on all aspects of our B+ tree implementation and because of our structure with the intermediate proposition we only had to use the validity preservation at the very end.

\subsubsection{The Use of Booleans}

Within our implementation we rely heavily on the use of $bool$ when comparing natural numbers (ie $beq_nat$, $ble_nat$ and $blt_nat$) and performing conjunctions (ie $band$). In retrospect this is unnecessary and convoluted the verification because of all the required conversions from $bool$ to $Prop$. It would likely have been simpler had we used $Either$ instead.

\subsubsection{Unverified Claims}

As mentioned in Section \ref{sec:ElementsOfACompleteProof}, we have limited the scope of this assignment to only cover the main theorem $insert\_search\_works$. This means that the two other theorems ($insert\_preserves\_elements$ and $insert\_preserves\_tree\_validity$) are completely unverified.

Because $insert\_search\_works$ relies on $insert\_preserves\_tree\_validity$ for it's final step, we can not claim to have completely verified this lemma. We have however individually verified the two theorems around our intermediate proposition.

%!TEX root = ../BPlusTree-report.tex
\section{Related Work}
\label{sec:RelatedWork}
Earlier work\,\cite{sexton2008reasoning} has reasoned about B+ trees using separation logic, but to our knowledge, no one has produced a complete proof of correctness for a B+ tree implementation\todo{Write about this}.

%\todo{Write about http://ynot.cs.harvard.edu/papers/popl10.pdf}

%\todo{Write about http://link.springer.com/content/pdf/10.1007%2F978-3-642-24690-6_14.pdf}


%!TEX root = ../BPlusTree-report.tex
\section{Conclusion}
\label{sec:Conclusion}
In this paper we presented a definition of B+ trees in Coq, and accounted for the performance and representational challenges connected to it. This included adding an extra zero key to the beginning of all nodes, as well as using the height of the given tree as a counter on which the insertion and search functions recursed. 
\paragraph{}
First we defined an inductive data type, called $valid\_bplustree~b~X$, for usage in various proofs related to the primary functions on B+ trees. Then we defined the three main theorems of a complete proof of correctness with regards to insertion and search. We decided to focus on $insert\_search\_works$, as we believed it to be the most significant of the three. As a consequence, we chose to leave $insert\_preserves\_tree\_validity$ and $insert\_preserves\_elements$ as future work.
\paragraph{}
To accommodate the need for reasoning about the search and insert functions independently, an intermediate proposition was defined. The $kv\_appears\_in\_tree$, $appears\_in\_tre$, $kv\_appears\_in\_kvl$, and $appears\_in\_kvl$ propositions were used to prove a list of facts that are informally described here:

\begin{itemize}
	\item If a key-value pair $(k, v)$ appears in a tree $t$ then $search~k~t = Some~(v)$
	\item If a key $k$ does not appear in a tree $t$ then $search~k~t = None$
	\item If the key of a key-value pair $(k, v)$ does not appear in a tree $t$, then performing $insert~k~v~t$ ensures that the pair appears in $t$.
\end{itemize} 

These facts were ultimately used to verify the correctness of $insert\allowbreak{}\_search\allowbreak{}\_works$, the main theorem of the paper stating that subsequent insertion of a key-value pair and search of the same key-value pair would find it again.

\input{sections/08.References.tex}
%!TEX root = ../BPlusTree-report.tex
\appendix
\label{sec:Appendix}
\section{Running the code}
The code for this project is hosted on Github at \url{https://github.com/nicolaidahl/BPlusTrees}. The Coq implementation resides in the ./code/ subdirectory. A Makefile is provided that will build the project. It has been tested with Coq 8.3pl5 and later.


\end{document}
