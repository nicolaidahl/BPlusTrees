%!TEX root = ./report.tex
\section{Conclusion}
\label{sec:conclusion}
As evaluated in Section \ref{sec:evaluation}, the requirements for a partial editor prototype set forward in Section \ref{subsec:requirements}, have been met. Furthermore, the correctness of the output has been verified with an account for threats to validity. Specifically, we have\ldots

\begin{itemize}
\item Shown that it is feasible to define and edit a partial IFC model for a specific domain using a concrete syntax, the Pipes DSL.
\item Developed an editor for the Pipes DSL with syntax highlighting and autocompletion using Xtext. A comparison between the Pipes DSL syntax and the IFC-EXPRESS syntax was shown in Figure \ref{fig:pipes_express_comparison}.
\item Changed metadata and edited the structure of a building model using the Pipes DSL. To the extent that all our tests run as expected, we believe that these changes happen correctly.
\item Been able to interface with a BIMServer instance to store building models and merge a construction model and a plumbing model into a coherent IFC model used by the workflows shown in Figures \ref{fig:IFC2PipesWorkflow} and \ref{fig:Pipes2IFCWorkflow}.
\item Presented ideas for projects that can improve upon the prototype and evolve the area of open source BIM software.
\end{itemize}

We believe that through the exploration of how model-driven development can be used with IFC, this pilot study has provided a valuable starting point for further research in this area, which the ITU research initiative Energy Futures can build upon. In conclusion, we find that editing of a partial IFC model defined by a concrete, real-world domain is feasible using the model-driven methodology.

\subsubsection{Acknowledgements} Special thanks to associate professor at Aalborg University Kaj A. Jørgensen for providing example IFC models as well as overall guidance in selecting a fitting IFC domain for our project. Furthermore, we thank construction management student at Copenhagen School of Design and Technology, Mathias Demant for providing example IFC models mainly used for testing.