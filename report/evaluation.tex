%!TEX root = ./report.tex
\section{Evaluation}
\label{sec:evaluation}
In this section we evaluate the solution, with reference to each of the requirements in Section \ref{subsec:requirements}. Furthermore, we discuss the methods used to verify the correctness of the prototype.

% Maybe: Discussion of applicability of product. What should be done in order to make this project usable in a real world scenario?

\subsection{Requirements Evaluation}
\label{subsec:requirements_evaluation}
\subsubsection{R1. Working with a Partial Model}
We are capable of extracting partial IFC models, and subsequently updating these by using the Pipes DSL editor. In Section \ref{subsec:problem_analysis} we presented the informal definition of the IFC subset that we extract, and in Figure \ref{fig:pipes_dsl_ecore_model} we depicted the corresponding Pipes DSL meta model. We focus on a simple partial model, but it would be possible to extend this if other IFC objects were to be added. We have ensured modularity by implementing the extraction process as a modular workflow component. Hence, the extraction process is verifiable as confirmed by the automated tests described in Section \ref{subsec:verification_methods}.

\subsubsection{R2. Correct Meta Model} As explained in Section \ref{subsec:ifc_meta_model}, the prototype uses the official ifcXML XSD to generate an Ecore meta model that is both correct and replaceable, in theory. However, there are certain serious drawbacks to this solution.

EMF does not seem to handle the large IFC meta model very well. The IFC meta model generated from the ifcXML XSD, was incomplete, with certain methods not being generated (eUnset methods for fields in some entity classes). Also, EMF generated methods that exceeded the maximum size allowed by the Java Development Kit (64kB)\footnote{JDK bug 4262078: \url{http://bugs.sun.com/view_bug.do?bug_id=4262078}}. While it is possible to work around such problems, it is not viable that the generated code has to be corrected.

One of the desirable features provided by using the generated meta model, was the availability of serialisers in EMF. However, these turned out to be too slow for working with realistic IFC models. A real-world model can easily result in load and save times of more than ten minutes. The EMF site lists some performance tips that should solve these problems\footnote{EMF Performance Tips: \url{http://www.eclipse.org/modeling/emf/docs/performance/EMFPerformanceTips.html}}\footnote{Performance and Extensibility with EMF: \url{http://www.slideshare.net/kenn.hussey/performance-and-extensibility-with-emf}}, but the improvement was minimal, if any. Thus, all of our tests have been carried out using smaller subsets of IFC models. 

\subsubsection{R3. Valid Model Transformations} 
To validate our transformations, both automated white box tests and black box test are used. The testing methods are described in detail in Section \ref{subsec:verification_methods}, and the results can be found in Appendices \ref{app:blackboxtests} and \ref{app:automatedtests}.

\subsubsection{R4. A Simple DSL} Section \ref{subsec:pipes_dsl} describes the benefits of the DSL implementation. It serves as a proof of concept that other subdomains of IFC can be edited in the same way. Being a modular component of the entire prototype, the textual Pipes DSL editor could be substituted with a visual one.

\subsubsection{R5. Structural Editing}
The prototype supports the addition and removal of both flow segments and openings, and thus the requirement is fulfilled.

\subsection{Verification Methods}
\label{subsec:verification_methods}
We have implemented automated test workflows for verifying that our transformation components work as desired. The IFCExtractor, IFC2PipesTransformer, and Pipes2IFCTransformer components (see Figure \ref{fig:Pipes2IFCWorkflow}) are the most interesting to validate, as these are the central components containing most of the transformation logic. Furthermore, black box tests of the entire prototype have been conducted with four input test buildings. These tests check that both the IFC model and the Pipes model is left in a valid state after running each workflow component. The results of the tests can be found in Appendices \ref{app:blackboxtests} and \ref{app:automatedtests}.

\subsection{Threats to validity}
\subsubsection{Internal Validity} When a prototype is primarily black box tested, it is natural to question the correctness of all the individual components. However, we argue that it is highly likely that any internal errors, undiscoverable by the black box tests, will either be discovered by the automated tests or result in program crashes, as the workflow components are highly modular with well-defined interfaces raising error flags if invalid input is provided.

Using EMF we have encountered a substantial amount of technical difficulties related only to the Eclipse IDE and its modelling tools. We identify these tools as an internal threat to the validity of the prototype, as our trust in them has been decreasing over the course of the project.

\subsubsection{External Validity} Retrieving building models for testing has been challenging. Given the fact that most modern visual BIM tools are very complex, we were unable to produce meaningful construction and plumbing models for testing, and had to rely on external contacts for providing test models (see Section \ref{sec:conclusion}). That means four example models were used for testing, which is considered sufficient, but not optimal. More tests with more example models must be performed to solidify the claim of general applicability of the prototype to all building models.







