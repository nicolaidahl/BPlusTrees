%!TEX root = ../BPlusTree-report.tex
\section{Introduction}
\label{sec:Introduction}
% Notes:
% Implement BPlusTree in Gallina
% What are we going to prove?
% We use Coq
In this paper, we discuss an implementation and formal proof of the B+ tree data structure in a pure functional language, Gallina, using the Coq interactive proof assistant. B+ trees are used in many commercial database and file systems\,\cite[p. 359]{ramakrishnan2003database}, and are characterized by their low height and high fanout, lessening the number of I/O operations needed when reading from disk. B+ trees are often described as ``... the best general data structure for database systems...''\,\cite[p. 84]{shasha2002database}.
\paragraph{}
We will present an inductively defined data type describing a B+ tree, and formulate the various propositions needed to reason about it. Based on this data type, we will implement $search$ and $insert$ functions in Gallina. We will then prove the correctness of $search$, and that the $insert$ function will insert elements into a B+ tree. Finally we will prove that after inserting an element with $insert$, it can be found with $search$. We will not present a complete proof of correctness for $insert$, but rather provide the groundwork that is needed for a complete proof.
\paragraph{}
The paper is structure as follows: In Section \ref{sec:Background} we explain the structure of B+ trees, and cover the theory behind the $search$ and $insert$ functions. Section \ref{sec:ProblemAnalysis} contains a problem analysis focused on the implementation of B+ trees in Gallina, as well as the definition of a valid B+ tree. The various strategies and difficulties when proving the main operations are covered in Section \ref{sec:ProofRealization}. Section \ref{sec:Evaluation} is an evaluation of our implementation and proofs, while Section \ref{sec:RelatedWork} discusses related work. Section \ref{sec:Conclusion} concludes the report.
