%!TEX root = ../BPlusTree-report.tex
\section{Introduction}
\label{sec:Introduction}
% Notes:
% Implement BPlusTree in Gallina
% What are we going to prove?
% We use Coq
In the following report we discuss the implementation and formal proof of the B+ tree data structure in a pure functional language, Gallina, using the Coq interactive proof assistant. B+ trees are used in many commercial database and file systems\,\cite[p. 359]{ramakrishnan2003database}, and are characterized by their low height and high fan-out, lessening the number of IO operations needed when reading from disk\,\cite[pp. 344]{ramakrishnan2003database}.
\paragraph{}
We will present an inductively defined data type describing a B+ tree, and formulate the various propositions needed to reason about it. Based on this data type, we will implement $search$ and $insert$ functions in Gallina. We will then prove the correctness of $search$, and we will prove that the $insert$ function can insert elements into a B+ tree. We will not present a complete proof of correctness for $insert$, but rather provide the groundwork that is needed.
\paragraph{Outline}
In Section \ref{sec:Background} we explain the structure of B+ trees, and cover the theory behind the $search$ and $insert$ functions. Section \ref{sec:ProblemAnalysis} contains a problem analysis focused on the implementation of B+ trees in Gallina. The various strategies and difficulties when proving the main operations are covered in Section \ref{sec:ProofRealization}. Finally, Section \ref{sec:Conclusion} concludes the report.
