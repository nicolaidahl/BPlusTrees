%!TEX root = ../BPlusTree-report.tex
\section{Related Work}
\label{sec:RelatedWork}
Sexton et al\,\cite{sexton2008reasoning} use an abstract machine to specify operations on B+ trees, and then use separation logic to reason about these operations, specifically the insert and range query operations. Their approach lets them reason locally about the subtree being modified, while the rest of the tree is invariant. This makes their proofs less complex, as they only have to take a small part of the tree into account. It should be noted that their work does not constitute a formal proof, but rather a rigorous informal proof.

\paragraph{}
Ernst et al\,\cite{ernst2011verification} use the TVLA tool to conduct shape analysis on B+ trees, automatically discharching many proof obligations related to the structure of B+ trees, such as proof of acyclicity, while using the KIV interactive proof assistant to prove move complicated obligations. They argue that automated shape analysis combined with interactive theorem proving is a great improvement over using one or the other alone.

\paragraph{}
Malecha et al\,\cite{malecha2010toward} \todo{Write about this}
