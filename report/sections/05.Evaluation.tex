%!TEX root = ../BPlusTree-report.tex
\section{Evaluation}
\label{sec:Evaluation}
Formally verifying a non-trivial data type like B+ trees has proven to be a difficult task. In this section we evaluate four central points regarding our work process and our InsertSearchWorks proof.

\subsubsection{Insert Into Leaf}
Our strategy for proving the $insert\_impl\_appears$ proof, was to prove the base case separately. The base case would be inserting into a tree with $height = 0$, or in other words a leaf. So the initial lemma we sat out to prove was $insert\_leaf\_impl\_appears$. But because we had not even attempted verifying the behaviors of inserting in nodes, we had some incorrect assumptions about what we should prove. One problem was that we didn't make the verified statements very strong, so when we later had to apply the lemma in $insert\_impl\_appears$ we had to establish a lot of context before we could apply the lemma. When we later sat out to verify $insert'$, we improved this aspect and verified a much stronger statement.

\subsubsection{Unnecessary Assumption}

Another shortcoming in our initial assumptions when verifying $insert\_leaf\_impl\_appears$ was that the key was not present before insertion. As a result, our verification of insertion into leafs has the requirement that $\lnot appears\_in\_kvl$ must hold. Because we rely on the verification for leafs when verifying insertion in trees, this assumption bubbles up and adds the requirement that $\lnot appears_in_tree$ must hold before inserting. In retrospect this assumption is unnecessary and results we have only a slightly weaker verification. We have only verified that inserting a new value works, not that overwriting an existing key works too. 

\subsubsection{Higher Priority for Validity Preservation Proof}

With a better overview we might have chosen to prove $insert\_preserves\_validity$ first. Our $insert\_search\_works$ proof relies on insertion preserving the validity, so without verification of this preservation we can not create a complete verification that inserted items can be found. The way we have structured our proofs, is only in the final lemma we require this preservation, so our two intermediate theorems are fully verified.

The reason why we chose to verify $insert\_search\_works$ first, that that we felt that a such verification was more telling about our actual implementation rather than proving that we maintained a inductively defined proposition. The verification of insertion and search relies on all aspects of our B+ tree implementation and because of our structure with the intermediate proposition we only had to use the validity preservation at the very end.

\subsubsection{The Use of Booleans}

Within our implementation we rely heavily on the use of $bool$ when comparing natural numbers (ie $beq_nat$, $ble_nat$ and $blt_nat$) and performing conjunctions (ie $band$). In retrospect this is unnecessary and convoluted the verification because of all the required conversions from $bool$ to $Prop$. It would likely have been simpler had we used $Either$ instead.

\subsubsection{Unverified Claims}

As mentioned in Section \ref{sec:ElementsOfACompleteProof}, we have limited the scope of this assignment to only cover the main theorem $insert\_search\_works$. This means that the two other theorems ($insert\_preserves\_elements$ and $insert\_preserves\_tree\_validity$) are completely unverified.

Because $insert\_search\_works$ relies on $insert\_preserves\_tree\_validity$ for it's final step, we can not claim to have completely verified this lemma. We have however individually verified the two theorems around our intermediate proposition.
