%!TEX root = ../BPlusTree-report.tex
\section{Evaluation}
\label{sec:Evaluation}
Formally verifying a non-trivial data type like B+ trees has proven to be a difficult task. In this section we evaluate four central points regarding our work process and our $insert\_search\_works$ proof.

\subsubsection{Insert Into Leaf}
Our strategy for proving the $insert\_impl\_appears$ proof was to prove the base case separately. The base case is inserting into a tree with $height = 0$, or in other words a leaf. So the initial lemma we set out to prove was $insert\_leaf\_impl\_appears$. But because we had not attempted to verify the behaviors of insertion in nodes, we had some incorrect assumptions about what we should prove. One problem was that we did not make the goal very strong, so when we later had to apply the lemma in $insert\_impl\_appears$ we had to establish a great deal in our context before we could apply the lemma. When we later set out to verify $insert'$, we improved this aspect and verified a much stronger statement.

\subsubsection{Unnecessary Assumption}
Another shortcoming in our initial assumptions when verifying $insert\_leaf\_impl\_appears$ was that the key was not present before insertion. As a result, our verification of insertion into leaves has the requirement that $\lnot appears\_in\_kvl$ must hold. Because we rely on the verification for leaves when verifying insertion in trees, this assumption bubbles up and adds the requirement that $\lnot appears\_in\_tree$ must hold before inserting. In retrospect this assumption is unnecessary, and results in a slightly weaker verification. We have only verified that inserting a new value works, not that overwriting an existing key works too. 

\subsubsection{Higher Priority for Validity Preservation Proof}

With a better overview we might have chosen to prove \textit{insert\allowbreak{}\_preserves\allowbreak{}\_validity} first. Our \textit{insert\allowbreak{}\_search\allowbreak{}\_works} proof relies on insertion preserving the validity, so without verification of this preservation we can not create a complete verification that inserted items can be found. The way we have structured our proofs means that it is only in the final lemma we require this preservation, so our two intermediate theorems are fully verified.

The reason why we chose to verify \textit{insert\_search\_works} first, is that that we felt that a such verification was more telling of our actual implementation rather than proving that we maintained a inductively defined proposition. The verification of $insert$ and $search$ relies on all aspects of our B+ tree implementation and because of our structure with the intermediate proposition, we only had to use the validity preservation at the very end.

\subsubsection{The Use of Booleans}

Within our implementation we rely heavily on the use of $bool$ when comparing natural numbers (i.e. $beq\_nat$, $ble\_nat$ and $blt\_nat$) and performing conjunctions (i.e. $andb$). In retrospect this is unnecessary and convoluted the verification because of all the required conversions from $bool$ to $Prop$. It would likely have been simpler had we used $Either$ instead.

\subsubsection{Unverified Claims}

As mentioned in Section \ref{sec:ElementsOfACompleteProof}, we have limited the scope of this assignment to only cover the main theorem $insert\_search\_works$. This means that the two other theorems ($insert\allowbreak{}\_preserves\allowbreak{}\_elements$ and $insert\allowbreak{}\_preserves\allowbreak{}\_tree\allowbreak{}\_validity$) are completely unverified.

Because $insert\_search\_works$ relies on $insert\_preserves\_tree\_validity$ for it's final step, we can not claim to have completely verified this theorem. We have however individually verified the two theorems around our intermediate proposition. Additionally, one of the lemmas used by $insert'\_implies\_appears$, $insert'\_overflow\_impl\_greater\_key'$ has a single admitted case. While this lemma appears solvable, it was not completely proven, due to time constraints.
